\section{Introduction}

Advances in computing capabilities are further improving the feasibility of
fast-running high-fidelity simulations of nuclear cores. Where current core
simulations require a series of homogenization procedures to model reactors on a
coarse-mesh in order to overcome memory and computer processing limitations
\cite{Smith1986303}, modern techniques aspire to provide solutions using
fully-detailed geometries with far fewer approximations. For instance, recent
research efforts improving the scalability and efficiency of Monte Carlo neutron
transport algorithms have resulted in very accurate solutions to the well-known
Hoogenboom-Martin problem \cite{HM_bench} with the MC21 Monte Carlo code
\cite{mc21_hoomartin2010} \cite{mc21_hoomartin2012}, with statistical
uncertainties approaching the 1\% pin-power accuracy criterion proposed by Smith
\cite{kordsmithchallenge} for full-core Monte Carlo analysis. Likewise, modern
deterministic approaches are targeting similar accuracy on some of the world's
largest supercomputers \cite{denovo_jaguar2012}. Indeed, the development of such
high-fidelity full-core modelling capabilities for \acp{LWR} is the stated goal
of several DOE projects such as \acs{CASL} (\acl{CASL}) \cite{casl_goals} and
\acs{CESAR} (\acl{CESAR}) \cite{cesar_goals}. However, there is a lack of
detailed and relevant benchmarks needed to validate these methods, and a more
complete benchmark that includes measured reactor data is presented here.

Nearly all non-proprietary benchmarks do not capture the detail of \acp{LWR}
needed to validate high-fidelity methods being developed today. For instance,
the OECD \ac{LWR} and \ac{PWR} reactor benchmark specifications
\cite{oecd_bench} mostly refer to simple lattice experiments, limited physics
testing configurations, and small test reactors. Whereas several full-core
\acs{LMFR} models are available (\acs{FFTF}, JOYO, etc.), \acp{LWR} are markedly
under represented. This is particularly true for full-core benchmarks of most
interest to the methods development and regulatory community: production
reactors similar to operating and planned commercial units. Some recent
publications come close to satisfying this need, but they ultimately fall short
either in scope or applicability. For instance, a 2011 \acs{EPRI} report
\cite{epri2011bench} provides reactivity and depletion data with several
benchmark specifications for \ac{PWR} assembly lattices. These benchmarks, while
using full-core simulations and measured data, take the approach of reducing the
benchmark to single-assembly calculations and do not provide detailed full-core
tests or measured reactor data.

A distinction should be noted between the kind of data-backed benchmark being
pursued here and the code-comparison benchmarks often used to evaluate methods.
For instance, several newer and widely-used \ac{LWR} benchmarking suites are not
backed by measured data, such as the C5G7 \acs{MOX} benchmarks \cite{c5g7}, the
Hoogenboom-Martin \ac{LWR} Monte Carlo benchmark \cite{mc21_hoomartin2010}, and
the approximate \ac{PWR} specification by Douglas et al.
\cite{Douglass20101384}. Instead, comparisons are based on results submitted by
many different parties using a variety of codes. Measured reactor data is
required for a credible validation.

This document introduces a new benchmark that addresses many of the shortcomings
of previous \ac{LWR} benchmarks by providing a highly-detailed \ac{PWR}
specification with two cycles of measured operational data that can be used to
validate high-fidelity core analysis methods.
